\documentclass{article}
\usepackage{a4wide}

\title{A beginners guide to the OPI in R}
\date{July 2012}
\author{Andrew Turpin aturpin@unimelb.edu.au}

\begin{document}
\maketitle

This document is written as an introduction to the OPI in R primarily
using the Octopus 900 for examples.

\section{Notation}

Code will be in typewriter font like {\tt this}.
Commands will be preceded by a {\tt >}. You should not enter the {\tt >} in your code: it is
the command prompt. Command continuations begin with a {\tt +}, you shouldn't enter these
either.

%%%%%%%%%%%%%%%%%%%%%%%%%%%%%%%%%%%%%%%%%%%%%%%%%%%%%%%%%%%%%%%%%%%%%%%%%%
\section{Installing}
\label{sec-install}

The OPI package will be in CRAN soon, I hope. In the meantime you have to install from source.
Inside R type the following.

\begin{verbatim}
> install.packages("c:\\OPI_v1.4.tar.gz", NULL, type="source")

> install.packages("c:\\OPIOctopus900_v1.1.tar.gz", NULL, type="source")
\end{verbatim}

Make sure the first argument to each call to the {\tt install.packages}
function is a valid path name to the package files.

Also make sure you have the {\tt rJava} package installed if you
want to use the Octopus 900 or the HEP. You need permission from
HAAG-STREIT to get the OPIOctopus900 package, and from Heidelberg
Engineering to get the Rhep package.




%%%%%%%%%%%%%%%%%%%%%%%%%%%%%%%%%%%%%%%%%%%%%%%%%%%%%%%%%%%%%%%%%%%%%%%%%%
\section{Your first stimulus}

Assuming you have the OPI package installed, and any other necessary
packages for controlling an external machine, you are ready to go.

First you must load the libraries with the {\tt library} or {\tt
require} functions as follows.

\begin{verbatim}
> require(OPI)
\end{verbatim}

If R complains about these with an error message, return to Section~\ref{sec-install}.

Now you must select which implementation of the OPI you want to
use: a simulation, or a machine. This is done with the {\tt chooseOPI}
function. You can get a list of the available implementations by
calling {\tt chooseOPI} with an argument of {\tt NULL}.

\begin{verbatim}
> chooseOpi(NULL)
        name          name          name          name          name 
 "Octopus900"       "SimNo"      "SimYes"   "SimHenson" "SimGaussian" 

[1] TRUE
\end{verbatim}

This shows us that there are three implementations available. Let us choose
the Octopus900 one.

\begin{verbatim}
> chooseOpi("Octopus900") 
Loading required package: OPIOctopus900 
[1] TRUE
\end{verbatim}

Note that choosing this implementation also loaded the required package
OPIOctopus900. If you do not have this package, then you cannot control the
Octopus 900; see Section~\ref{sec-install}.

Once an OPI implementation is chosen, you need to execute opiInitialize.
Depending on your implementation, different arguments are required for this
function. You can get help on any function by typing a question mark and
then the function name. For example:

\begin{verbatim}
> ?chooseOpi
> ?opiInitialize
\end{verbatim}

Note that all arguments to opi functions must be named.

For the Octopus 900, two folder names and the eye to be tested is required.
\begin{verbatim}
> opiInitialize(eyeSuiteJarLocation="c:\Program Files\EyeSuite\",
+ eyeSuiteSettingsLocation="c:\Documents and Settings\All Users\HAAG STREIT\EyeSuite\",
+ eye="right")
NULL
\end{verbatim}
The first folder is the location of the EyeSuite software. The folder should contain files
named {\tt HS$\ast$.jar}.
The second folder is the location of the EyeSuite settings.
This command should return NULL if successful.

Note that if the Octopus 900 has just been turned on, there will
be a short delay before this function returns while the machine is
calibrated. You should hear whirring and clicking noises from the
machine.  Also make sure that EyeSuite is not trying to use the
Octopus 900 at the same time (close EyeSuite).

Now you can present a stimuli, and get a response. A stimuli is specified with a {\tt
opiStaticStimulus} object, which is just a list of named values with a class of {\tt
opiStaticStimulus}. For example, 

\begin{verbatim}
    # Specify a 17 dB stimulus at $(9,9)$ for 200ms, and 
    # wait for 1500ms to see if there is a button press.
>  s <- list(x=9, y=9, level=dbTocd(17), size=0.43, duration=200, 
+            responseWindow=1500)
>  class(s) <- "opiStaticStimulus"
\end{verbatim}

The full documentation of {\tt opiStaticStimulus} is in the
OPI Standard document, and in the help files (remember {\tt
?opiStaticStimulus}).

Note the use of the function {\tt dbTocd} for converting db values
in to cd/m$^2$, which is what an {\tt opiStaticStimulus} should
specify the stimulus level in. There is also a {\tt cdTodb} function.

Often it is useful to have a helper function to construct the {\tt
opiStaticStimulus} for you, to save lots of typing. For example,
this function that takes in a dB value, and builds a stimulus at
location (9,9) for that dB value.

\begin{verbatim}
>   makeStim <- function(db) {
+        s <- list(x=9, y=9, level=dbTocd(db), size=0.43, color="white",
+                 duration=200, responseWindow=1500)
+        class(s) <- "opiStaticStimulus"
+
+        return(s)
+    }
\end{verbatim}            

Now that you have a stimulus, you can present it with {\tt opiPresent} as follows.
\begin{verbatim}
> opiPresent(stim=makeStim(0), nextStim=makeStim(0))
\end{verbatim}            
This will only work if you have successfully run {\tt opiInitialize}.
Note that {\tt opiPresent} takes two stimulus arguments: the first is the one to present, and
the second is the one that will be presented next. This allows the projector to be moved to
the next location during the response window of the current stimulus.
In our example we simply made them at the same location, but usually this will not be the case.

Finally, you should close the OPI implementation. This is especially important for the Octopus
900 as EyeSuite will not function if the OPI has hold of the machine.

\begin{verbatim}
> opiClose()
\end{verbatim}



%%%%%%%%%%%%%%%%%%%%%%%%%%%%%%%%%%%%%%%%%%%%%%%%%%%%%%%%%%%%%%%%%%%%%%%%%%
\section{Changing backgrounds and colors}

On the Octopus 900, {\tt opiPresent} ignores the {\tt color} field in the {\tt
opiStaticStimulus} list, and only presents stimuli of the current
set color.
{\tt opiInitialize} sets the background and
stimulus to the standard white-on-white settings.

If you want to change the stimulus color, or the background from
white to yellow, you must use {\tt opiSetBackgound}. There are 3
possible stimulus colors, and two possible background colors, so
there are 6 combinations, as shown in Table~\ref{tab-met-col}.
You can choose one of these with the {\tt opiSetBackgound} function, specifying the {\tt color}
argument as one of the constants in the final column.
For example,
\begin{verbatim}
> opiSetBackgound(color=opi.O900.MET_COL_BW)
\end{verbatim}

\begin{table}
\centering
\begin{tabular}{ccc}
\hline
Stimulus & Background & OPI Constant \\
\hline
White    & White & opi.O900.MET\_COL\_WW \\
Blue     & White & opi.O900.MET\_COL\_BW \\
Red      & White & opi.O900.MET\_COL\_RW \\
White    & Yellow & opi.O900.MET\_COL\_WY \\
Blue     & Yellow & opi.O900.MET\_COL\_BY \\
Red      & Yellow & opi.O900.MET\_COL\_RY \\
\hline
\end{tabular}
\caption{\label{tab-met-col}Six possible color combinations on the Octopus 900}
\end{table}

You can also change the luminance of the background with the {\tt lum} argument of the {\tt
opiSetBackgound} function. It is in cd/m$^2$.
For example,
\begin{verbatim}
> opiSetBackgound(lum=0)    # turn the background light off
> opiSetBackgound(lum=100)  # turn the background to 100 cd/m2
\end{verbatim}
See {\tt ?opiSetBackgound} for more information.

%%%%%%%%%%%%%%%%%%%%%%%%%%%%%%%%%%%%%%%%%%%%%%%%%%%%%%%%%%%%%%%%%%%%%%%%%%
\section{Changing fixation markers}

The Octopus 900, has 3 possible fixation targets
\begin{verbatim}
opi.O900.FIX_RING
opi.O900.FIX_CROSS
opi.O900.FIX_CENTRE
\end{verbatim}
and each can be the {\tt fixation} argument of {\tt opiSetBackgound}. For example, 
\begin{verbatim}
opiSetBackgound(fixation=opi.O900.FIX_RING)
\end{verbatim}

You can also set the intensity of the fixation marker as a percentage from 0 to 100 using the
{\tt fixIntensity} argument. 
For example,
\begin{verbatim}
> opiSetBackgound(fixIntensity=0)    # turn fixation marker off
> opiSetBackgound(fixIntensity=100)  # fixation marker is brightest
\end{verbatim}

%%%%%%%%%%%%%%%%%%%%%%%%%%%%%%%%%%%%%%%%%%%%%%%%%%%%%%%%%%%%%%%%%%%%%%%%%%
\section{Built-In Test Procedures}

The OPI has Full Threshold (FT).
See {\tt ?FT}


%%%%%%%%%%%%%%%%%%%%%%%%%%%%%%%%%%%%%%%%%%%%%%%%%%%%%%%%%%%%%%%%%%%%%%%%%%
\end{document}



