\documentclass{article}

\title{A beginners guide to the OPI in R}
\date{July 2012}
\author{Andrew Turpin aturpin@unimelb.edu.au}

\begin{document}
\maketitle

\section{Notation}

Code will be in typewriter font like {\tt this}.
Commands will be preceeded by a {\tt >}. You should not enter the {\tt >} in your code: it is
the command prompt.

%%%%%%%%%%%%%%%%%%%%%%%%%%%%%%%%%%%%%%%%%%%%%%%%%%%%%%%%%%%%%%%%%%%%%%%%%%
\section{Installing}
\label{sec-install}

Inside R type the following.

\begin{verbatim}
> install.packages("c:\\OPI_v1.1.tar.gz", NULL, type="source")

> install.packages("c:\\OPIOctopus900_v1.0.tar.gz", NULL, type="source")
\end{verbatim}

Make sure the first argument to each call to the {\tt install.packages}
function is a valid path name to the package files.

Also make sure you have the {\tt rJava} package installed if you
want to use the Octopus 900 or the HEP. You need permission from
HAAG-STREIT to get the OPIOctopus900 package, and from Heidelberg
Engineering to get the rHep package.




%%%%%%%%%%%%%%%%%%%%%%%%%%%%%%%%%%%%%%%%%%%%%%%%%%%%%%%%%%%%%%%%%%%%%%%%%%
\section{First try}

Assuming you have the OPI package installed, and any other nevessary
packages for controlling an external machine, you are ready to go.

First you must load the libraries with the {\tt library} or {\tt
require} functions as follows.

\begin{verbatim}
> require(OPI)
\end{verbatim}

If R complains about these with an error message, return to Section~\ref{sec-install}.

Now you must select which implementation of the OPI you want to
use: a simulation, or a machine. This is done with the {\tt chooseOPI}
function. You can get a list of the available implementations by
calling {\tt chooseOPI} with an argument of {\tt NULL}.

\begin{verbatim}
> chooseOpi(NULL)
        name         name           name    name 
"Octopus900"   "SimHenson" "SimGaussian"   "HEP"
[1] TRUE
\end{verbatim}

This shows us that there are three implementations available. Let’s choose
the Octopus900 one.

\begin{verbatim}
> chooseOpi("Octopus900") 
Loading required package: OPIOctopus900 
[1] TRUE
\end{verbatim}

Note that choosing this implementation also loaded the required package
OPIOctopus900. If you do not have this package, then you cannot control the
Octopus 900. See Section~\ref{sec-install}.

Once an OPI implementation is chosen, you need to execute opiInitialize.
Depending on your implementation, different arguments are required for this
function. You can get help on any function by typing a question mark and
then the function name. For example:

\begin{verbatim}
> ?chooseOpi
> ?opiInitialize
\end{verbatim}

For the Octopus 900, two folder names and the eye to be tested is required.
\begin{verbatim}
> opiInitialize("c:\Program Files\EyeSuite\",
+ "c:\Documents and Settings\All Users\HAAG STREIT\EyeSuite\",
+ "right")
NULL
\end{verbatim}
The first folder is the location of the EyeSuite software. The folder should contain files
named {\tt HS$\ast$.jar}.
The second folder is the location of the EyeSuite settings.
This command should return NULL if successful.

Note that if the Octopus 900 has just been turned on, there will
be a short delay before this function returns while the machine is
calibrated. You should hear whirring and clicking noises from the machine.

Now you can present a stimuli, and get a response. A stimuli is specified with a {\tt
opiStaticStimulus} object, which is just a list of named values.

[[ TODO - add more here. ]]

\begin{verbatim}
>   makeStim <- function(db) {
+        s <- list(x=9, y=9, level=dbTocd(db), size=0.43, color="white",
+                 duration=200, responseWindow=1500)
+        class(s) <- "opiStaticStimulus"
+
+        return(s)
+    }

> opiPresent(makeStim(0), makeStim(0))
\end{verbatim}            
                          
[[ TODO - add more here. ]]

Finally, you chould close the OPI implementation. This is especially important for the Octopus
900 as EyeSuite will not function if the OPI has hold of the machine.

\begin{verbatim}
> opiClose()
\end{verbatim}
%%%%%%%%%%%%%%%%%%%%%%%%%%%%%%%%%%%%%%%%%%%%%%%%%%%%%%%%%%%%%%%%%%%%%%%%%%
\end{document}



